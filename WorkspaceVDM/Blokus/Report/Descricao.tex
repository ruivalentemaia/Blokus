Este trabalho prático está inserido na disciplina de Métodos Formais e Engenharia de Software e tem como principal objetivo a elaboração, documentação e teste de um modelo formal executável de um sistema de software em VDM++. Com este objetivo foi usado como IDE o Overture para a realização da especificação formal do trabalho, foi usado o VDM++ Toolbox para converter a especificação em código-fonte Java e o Eclipse para criar a função main que corre o jogo.
O trabalho escolhido pelo grupo foi o Blokus, que é um jogo de tabuleiro para entre 2 jogadores e é jogado num tabuleiro de dimensões 20x20 e onde as peças de cada jogador podem assumir um total de 21 formas diferentes, assim como sofrer inversões e rotações.  Cada peça pode ocupar entre 1 a 5 casas do tabuleiro – sendo que o tabuleiro possui um total de 400 casas – e pode ter 4 cores diferentes (azul, amarelo, vermelho e verde).
A primeira peça jogada por cada jogador tem de ser colocada num dos cantos do tabuleiro, sendo que as peças seguintes que cada jogador joga só podem ser colocadas com uma das casas integrantes da peça a tocar com o canto num dos cantos das peças já inseridas no tabuleiro. O jogo termina quando nenhum dos jogadores consegue colocar mais nenhuma das suas peças no tabuleiro, sendo que a pontuação é atribuída quando tal acontece, sendo que cada casa do tabuleiro ocupada pelas peças de cada jogador incrementa um ponto nas suas pontuações.
