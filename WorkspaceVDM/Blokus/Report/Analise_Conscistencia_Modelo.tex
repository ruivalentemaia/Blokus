Segundo as regras do jogo Blokus [1], observa-se que todas as regras de jogo estão devidamente implementadas, exceptuando a possibilidade de extender o jogo de 2 até 4 jogadores. Contudo, optou-se por não o fazer na medida em que o Blokus, a ser jogado em modo offline – e não online, como a maioria das versões do mesmo que existem o são – tornar-se-ia aborrecido, demasiado extensivo e pouco prático para mais do que 2 jogadores.
De qualquer modo, o modelo em VDM++ que foi criado para implementar a lógica do jogo, cumpre todas as restantes regras do Blokus. Na medida em que este é um jogo simples, optou-se por criar apenas 2 classes: a classe Game, que implementa as validações necessárias para o jogo funcionar de acordo com as regras especificadas e a classe Tile, que modela as peças possíveis de serem jogadas assim como implementa a rotação e inversão de cada uma delas, de forma a que o jogador as possa manipular da forma que pretende.
Na classe Game, 
<incluir pequeno parágrafo a explicar o que faz a classe Game>
<incluir pequeno parágrafo a explicar o que faz a classe Tile>
Dado o elevado grau de verificação incluído no código e o pormenor e a profundidade com que todas as situações de jogo são analisadas, podemos concluir que este modelo é consistente e que pode ser devidamente jogado sem que sejam omitidos elementos importantes em toda a mecânica de jogo, de forma a dar prazer aos utilizadores que o experimentem.
