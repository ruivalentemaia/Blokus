\section{RES1}
\subsection{Classe}
Game
\subsection{Descrição}
Ordem de jogo para 2 jogadores: Azul[B] - Amarelo[Y] - Azul[B]
\\Ordem de jogo para 3 jogadores: Azul[B] - Amarelo[Y] - Vermelho[R] - Azul[B]
\\Ordem de jogo para 4 jogadores: Azul[B] - Amarelo[Y] - Vermelho[R] - Verde[G] - Azul[B]
\subsection{Especificação VDM++}
\begin{vdm_al}
public changeTurn: () ==> () 
  changeTurn() == ( 
      if turn = <blue> then 
        turn := <yellow> 
      else if nPlayers = 2 then 
          turn := <blue> 
        else if nPlayers = 3 then ( 
          if turn = <yellow> then 
            turn := <red> 
        else 
          turn := <blue> 
          ) 
      else if turn = <yellow> then 
        turn := <red> 
      else if turn = <red> then 
        turn := <green> 
    else 
      turn := <blue>                
  ) 
  post turn <> turn~;
\end{vdm_al}

\section{RES2}
\subsection{Classe}
Game
\subsection{Descrição}
A primeira peça tem de ser colocada num dos 4 cantos do tabuleiro.
\subsection{Especificação VDM++}
\begin{vdm_al}
public placeFirstTile: Tile * Utilities`Color * Utilities`Pos ==> bool 
  placeFirstTile(t, c, d) == ( 
     
      for all p in set elems t.getShape() do (   
        board := board ++ {mk_Utilities`Pos(p.x+d.x,p.y+d.y) |-> c}; 
      ); 

    return true; 
  ) 
  pre validCorner(t, d) 
  post updateScore(t,c);

  public validCorner: Tile * Utilities`Pos ==> bool
    validCorner(t, d) == (
      
  dcl newShape : seq of Utilities`Pos := [];

  --deslocar a peca
  for all p in set elems t.getShape() do (
    dcl newPosition: Utilities`Pos := mk_Utilities`Pos(p.x+d.x, p.y+d.y);
        
        if board(newPosition) <> <empty> then
          return false;
          
      newShape := newShape ^ [newPosition];
      );
      
    --verificar se uma das posicoes e um corner
    
      if (card ( (elems newShape) inter (elems boardCorners) ) ) = 1 then
        return true;

      return false
    );
\end{vdm_al}

\section{RES3}
\subsection{Classe}
Game
\subsection{Descrição}
Todas as peças seguintes têm de tocar canto-a-canto pelo menos 1 peça da mesma cor.
\subsection{Especificação VDM++}
\begin{vdm_al}
  for all p in set elems t.getShape() do (
  <...>
    if p in set elems t.getCorners() then (
      dcl upperLeft: Utilities`Pos := mk_Utilities`Pos(newPosition.x,newPosition.y);
      dcl upperRight: Utilities`Pos := mk_Utilities`Pos(newPosition.x,newPosition.y);              
      dcl bottomLeft: Utilities`Pos := mk_Utilities`Pos(newPosition.x,newPosition.y);
      dcl bottomRight: Utilities`Pos := mk_Utilities`Pos(newPosition.x,newPosition.y);
  
      if ( (newPosition.x-1) >= 0 and (newPosition.y-1) >= 0) then
        upperLeft := mk_Utilities`Pos(newPosition.x-1,newPosition.y-1);
  
      if ( (newPosition.x+1)  < Utilities`boardSize and (newPosition.y-1) >= 0) then
        upperRight := mk_Utilities`Pos(newPosition.x+1,newPosition.y-1);
    
      if ( (newPosition.x-1) >= 0 and (newPosition.y+1) < Utilities`boardSize) then              
        bottomLeft := mk_Utilities`Pos(newPosition.x-1,newPosition.y+1);
    
      if ( (newPosition.x+1)  < Utilities`boardSize
    and (newPosition.y+1)  < Utilities`boardSize) then
        bottomRight := mk_Utilities`Pos(newPosition.x+1,newPosition.y+1);
  
      if board(upperLeft) = c or board(upperRight) = c or board(bottomLeft) = c or board(bottomRight) = c then
        cornerFound := true
      );
    );

    if cornerFound then
      return true;
\end{vdm_al}

\section{RES4}
\subsection{Classe}
Game
\subsection{Descrição}
As arestas de uma peça não podem tocar nas arestas de outra peça da mesma cor.
\subsection{Especificação VDM++}
\begin{vdm_al}
if board(newPosition) <> <empty> or board(right) = c or board(left) = c 
  or board(top) = c or board(bottom) = c then 
return false;
\end{vdm_al}

\section{RES5}
\subsection{Classe}
Game
\subsection{Descrição}
O jogador só pode colocar uma em posições vazias, isto é, as peças não se podem sobrepôr.
\subsection{Especificação VDM++}
\begin{vdm_al}
--verificar se as posicoes do tabuleiro estao vazias
for all p in set elems t.getShape() do (

  dcl newPosition: Utilities`Pos := mk_Utilities`Pos(p.x+d.x,p.y+d.y);
  dcl left: Utilities`Pos := mk_Utilities`Pos(newPosition.x,newPosition.y);
  dcl right: Utilities`Pos := mk_Utilities`Pos(newPosition.x,newPosition.y);              
  dcl top: Utilities`Pos := mk_Utilities`Pos(newPosition.x,newPosition.y);
  dcl bottom: Utilities`Pos := mk_Utilities`Pos(newPosition.x,newPosition.y);
  
  if ((newPosition.x-1) >= 0) then
    left := mk_Utilities`Pos(newPosition.x-1,newPosition.y);
  
  if ((newPosition.x+1) < Utilities`boardSize) then  
    right := mk_Utilities`Pos(newPosition.x+1,newPosition.y);
    
  if ((newPosition.y+1) < Utilities`boardSize) then              
    top := mk_Utilities`Pos(newPosition.x,newPosition.y+1);
    
  if ((newPosition.y-1) >= 0) then
    bottom := mk_Utilities`Pos(newPosition.x,newPosition.y-1);

  -- se posicao nao estiver vazia ou laterais tiver outra peca da mesma cor entao erro
  if board(newPosition) <> <empty> or board(right) = c or board(left) = c or board(top) = c or board(bottom) = c then
    return false;
\end{vdm_al}
